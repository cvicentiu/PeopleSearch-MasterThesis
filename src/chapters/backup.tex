\chapter{Introduction}
\label{chapter:intro}

The internet is now a major source of information. Billions of queries are performed
each day within search engines \cite{search-stats}. Out of all these queries, between 11 and 17\%
include a person name and 4\% contain only a person name \cite{weps2-eval}\cite{weps3-eval}.
The results returned by search engines is comprised of millions of documents 
that feature those particular key-words. This leads to the number of results 
in the millions. Extracting the correct information from the documents returned 
is a very difficult process for a human. This is particularly the case when it 
comes to people search. There is a huge overlap in names. The ratio of identical 
names to different people is around 1 to 10000. Detecting which person the document 
refers to is not a trivial task, not for humans, nor for machines.

The information comes from a noisy environment, the Internet. Most of the data
is not structured the same way and there are frequent omissions or mistakes.
This further complicates the process of extracting the correct information.
The recent popularity of social networks lessens this problem to a certain
extent. \cite{social-networks} The most noteworthy aspect to take into account is 
the structured data that social network websites present. \cite{social-networks-scraping}
Apart from the regular html template that the website posseses, which makes writing
web scrapers easier, it also offers a reasonably strong guarantee that the labeled
data is also correct. An example of this is the first name of the person's profile.
It is usually present in a preset DOM node and will in almost all cases contain
the correct spelling of the person's first name.

Given the previously mentioned advantages for social networks, we have chosen
to not focus on a general use case of disambiguating random web pages. Instead
we have made use of the social networks reliable structure and have procured
a dataset which features pages from the most popular social networking sites,
such as Facebook, Twitter, Google+, but also from popular professional networks
such as LinkedIn, GitHub, Quora, etc.

The dataset represents a snapshot of a subset of profiles from the main
social network sites, fetched in the first half of 2015. The subset contains
approximately half a million profile pages. The structure of the dataset
is described in more detail in \labelindexref{Section}{sub-sec:dataset-desc}.

The goal of the project is to correctly identify all the cliques present in the
current dataset. In order to detect all entities within the graph, we train a classifier
which marks which profiles (nodes) within the graph belong to the same entity,
by identifying which profiles should be linked together.

\section{Project Description}
\label{sec:proj}

\subsection{Project Scope}
\label{sub-sec:proj-scope}

This thesis presents the \textbf{\project}.

This is an example of a footnote \footnote{\url{www.google.com}}. You can see here a reference to \labelindexref{Section}{sub-sec:proj-objectives}.

Here we have defined the CS abbreviation.\abbrev{CS}{Computer Science} and the UPB abbreviation.\abbrev{UPB}{University Politehnica of Bucharest}

The main scope of this project is to qualify xLuna for use in critical systems.


Lorem ipsum dolor sit amet, consectetur adipiscing elit. Aenean aliquam lectus vel orci malesuada accumsan. Sed lacinia egestas tortor, eget tristiqu dolor congue sit amet. Curabitur ut nisl a nisi consequat mollis sit amet quis nisl. Vestibulum hendrerit velit at odio sodales pretium. Nam quis tortor sed ante varius sodales. Etiam lacus arcu, placerat sed laoreet a, facilisis sed nunc. Nam gravida fringilla ligula, eu congue lorem feugiat eu.

Lorem ipsum dolor sit amet, consectetur adipiscing elit. Aenean aliquam lectus vel orci malesuada accumsan. Sed lacinia egestas tortor, eget tristiqu dolor congue sit amet. Curabitur ut nisl a nisi consequat mollis sit amet quis nisl. Vestibulum hendrerit velit at odio sodales pretium. Nam quis tortor sed ante varius sodales. Etiam lacus arcu, placerat sed laoreet a, facilisis sed nunc. Nam gravida fringilla ligula, eu congue lorem feugiat eu.


\subsection{Project Objectives}
\label{sub-sec:proj-objectives}

We have now included \labelindexref{Figure}{img:report-framework}.

\fig[scale=0.5]{src/img/reporting-framework.pdf}{img:report-framework}{Reporting Framework}


Lorem ipsum dolor sit amet, consectetur adipiscing elit. Aenean aliquam lectus vel orci malesuada accumsan. Sed lacinia egestas tortor, eget tristiqu dolor congue sit amet. Curabitur ut nisl a nisi consequat mollis sit amet quis nisl. Vestibulum hendrerit velit at odio sodales pretium. Nam quis tortor sed ante varius sodales. Etiam lacus arcu, placerat sed laoreet a, facilisis sed nunc. Nam gravida fringilla ligula, eu congue lorem feugiat eu.

We can also have citations like \cite{iso-odf}.

\subsection{Dataset Description}
\label{sub-sec:dataset-desc}

\todo{Here be dataset description}

\subsection{Related Work}

Lorem ipsum dolor sit amet, consectetur adipiscing elit. Aenean aliquam lectus vel orci malesuada accumsan. Sed lacinia egestas tortor, eget tristiqu dolor congue sit amet. Curabitur ut nisl a nisi consequat mollis sit amet quis nisl. Vestibulum hendrerit velit at odio sodales pretium. Nam quis tortor sed ante varius sodales. Etiam lacus arcu, placerat sed laoreet a, facilisis sed nunc. Nam gravida fringilla ligula, eu congue lorem feugiat eu.


Lorem ipsum dolor sit amet, consectetur adipiscing elit. Aenean aliquam lectus vel orci malesuada accumsan. Sed lacinia egestas tortor, eget tristiqu dolor congue sit amet. Curabitur ut nisl a nisi consequat mollis sit amet quis nisl. Vestibulum hendrerit velit at odio sodales pretium. Nam quis tortor sed ante varius sodales. Etiam lacus arcu, placerat sed laoreet a, facilisis sed nunc. Nam gravida fringilla ligula, eu congue lorem feugiat eu.


Lorem ipsum dolor sit amet, consectetur adipiscing elit. Aenean aliquam lectus vel orci malesuada accumsan. Sed lacinia egestas tortor, eget tristiqu dolor congue sit amet. Curabitur ut nisl a nisi consequat mollis sit amet quis nisl. Vestibulum hendrerit velit at odio sodales pretium. Nam quis tortor sed ante varius sodales. Etiam lacus arcu, placerat sed laoreet a, facilisis sed nunc. Nam gravida fringilla ligula, eu congue lorem feugiat eu.

We are now discussing the \textbf{Ultimate answer to all knowledge}.
This line is particularly important it also adds an index entry for \textit{Ultimate answer to all knowledge}.\index{Ultimate answer to all knowledge}

\subsection{Demo listings}

We can also include listings like the following:

% Inline Listing example
\lstset{language=make,caption=Application Makefile,label=lst:app-make}
\begin{lstlisting}
CSRCS = app.c
SRC_DIR =..
include $(SRC_DIR)/config/application.cfg
\end{lstlisting}

Listings can also be referenced. References don't have to include chapter/table/figure numbers... so we can have hyperlinks \labelref{like this}{lst:makefile-test}.

\subsection{Tables}

We can also have tables... like \labelindexref{Table}{table:reports}.

\begin{center}
\begin{table}[htb]
  \caption{Generated reports - associated Makefile targets and scripts}
  \begin{tabular}{l*{6}{c}r}
    Generated report & Makefile target & Script \\
    \hline
    Full Test Specification & full_spec & generate_all_spec.py  \\
    Test Report & test_report & generate_report.py  \\
    Requirements Coverage & requirements_coverage &
    generate_requirements_coverage.py   \\
    API Coverage & api_coverage & generate_api_coverage.py  \\
  \end{tabular}
  \label{table:reports}
\end{table}
\end{center}
