\chapter{Related Work}
\label{chapter:project-desc}

There have been a number of attempts performed in regards to web people search.
Most relevant solutions have been proposed as part of the Web People Search competition,
\cite{weps3-eval}\cite{weps2-eval}. The competition focuses on a more general
problem, which is to identify web pages that describe a single individual. The
scoring of the competition is based on the quality of the clusters generated.

The dataset that WePS provides is comprised of a set of HTML web pages.
The pages contain the first 200 \cite{weps3-eval} search results returned by a search engine, given
a specific string query. This unfortunately leads to many web pages with
little to no content, as well as potentially intermediate redirect pages.
In other words the dataset is quite noisy and it is very difficult to perform
a complete clusterization of all the inputs, simply due to a lack of data within
the web pages.

\section{Person Identification (Entity Deduplication)}
\label{sec:person-id}
This represents the main task brought forward by the WePS \cite{weps2-eval} competition.
It consists of generating the relevant clusters of pages, with each cluster corresponding
to a single entity. There are a number of possible metrics available to judge the accuracy
of the classification. Some metrics provide a better gauge for correctness and utility of
the classification than others. Such metrics include:

\begin{itemize}
    \item
      Purity and Inverse Purity metrics. 
      A high purity score implies that a cluster has documents referring to one
      particular individual, while a high inverse purity metric implies that as
      many persons as possible are assigned to clusters.
    \item
      BCubed Precision and Recall metrics
      The percentage of documents referring to a single person within a cluser
      represents precision, while the recall metric refers to the number of people assigned
      to the correct cluster.
\end{itemize}

As was shown in the evaluation of the competition \cite{weps2-eval}, the first
metric is prone to errors. It is possible to obtain high scores for both metrics.
By adding each document to its own cluster as well as creating a cluster that contains
all documents duplicated, we achieve a perfect inverse purity score, as each
person is assigned to a cluster. The single "big" cluster has a low purity score, however the
other single clusters provide a skewing factory with a high purity score. The
classification is meaningless, but yet it provides greater than average scores.

The second metric is more reliable as it is very difficult to obtain both a high
precision and recall rate at the same time. During the challenge, teams either swayed
towards a higher precision with less recall, or vice versa. This problem is
also apparent in our proposed solution, given a limited feature set. However,
we are able to improve our precision score while maintaining adequate recall
rates with the addition of discriminant features.

\section{Best WePS-3 Result}
\label{sub-sub-sec:best-weps-result}
The best approach \cite{weps-best} for the final WePS-3 competition provides a
balanced result of around 60\% \cite{weps3-eval} precision and recall. The results
are, however, skewed by the test set size as well as the training set size. Due
to the size of the dataset, it is impossible to train a classifier that works for
a general use case and thus its performance often gets tuned on the dataset's domain,
more so than the whole problem.

The best approach makes use of a number of techniques to produce the final
clusterization. A few of these techniques are:
\begin {itemize}
    \item
        Feature extraction from the documents using the following models:
    \begin {itemize}
        \item {Wikipedia Concepts as the basis for feature detection and extraction.
               In short, it matches phrases to wikipedia article titles.}
        \item {Bag-of-words approach, similar to many conventional approaches.}
        \item {Named Entity Recognition that splits words into various categories
               related to entities.}
   \end{itemize}
   \item
       Merge all the features discovered in a document into an associated feature-vector.
   \item
       Attach a weight to each feature, according to a feature weighting model.
       {\em This approach is what clearly separates the submission from the rest.\/}

   \item
       Compute similarity between documents using cosine similarity and overlap similarity.
\end{itemize}
The most important aspect of the classification idea is bound to the feature
weighting model, that is comprised of two separate models. One is related to the
feature relevance to the search query and is based on the proximity of the feature
to the sentence that matches the query. The second model is related to the content
relevance of the feature. 

The content relevance of the feature is important due to the following reasons:
If the query name is the main topic of the page (say, the home page of a person),
all or most of the features are related to that query. In that case, their relevance
to the page content is similar to their relevance to the query name. 
If the query name is not the main topic of the content, for example, a query name
“David Beckham” mentioned in some news article talking about soccer games, though
the query names like “David Beckham” are not the main topic of the articles, they
tend to refer to the same person in the text with words like “soccer”, and “game” etc.
In order to compute content relevance, a machine learning model is used. The machine
learning model uses both content, as well as some page structural features:
\begin {enumerate}
\item Structural characteristics of the feature: presence in the title, in the meta keywords,
table headers, table body, lists, etc;
\item Visual characteristics such as font size, frequency in the visual title, bolding, italics.
\item General characteristics such as the feature category, its domain, language, length.
\item Non-structural characteristics, including TFIDF score.
\item Corpus level characteristics, including document frequency, dominant category etc.
\end {enumerate}

Clusterization is performed based on the feature vectors, using the similarity of the
documents, as described above. The clusterization algorithm is Hierarchical Agglomerative
Clustering. Final performance is improved when using both weighting models combined,
as opposed to a single one.

\section{Exploiting web querying for web people search}
A pairwise approach for solving the WePS 2 challenge is proposed by \cite{nuray2009exploiting}.
The authors propose a 2 step process. One is to first compute the probability
that two pages are refering to the same person. The second step is to perform
a clusterization by joining pairs that feature a sufficiently high probability
of representing the same person.

The probability function is computed by joining multiple features. Some of the more
relevant features are:
\begin {itemize}
\item
   {\em Named-Entity-Based similarity \/}. Named entities include person, organization and location names,
   extracted with a named entity recognizer. After the named entities are extracted,
   the similarity is computed by making use of a standard TF/IDF weighting scheme.
\item
   {\em Middle Name Initial Dissimilarity \/}. The existance of a difference between
   middle name initials is evidence that the similar names are not refering to the same
   individual.
\item
   {\em Hyperlinks-Based Similarity \/}. Any links between very similar domain names,
   such as those that feature the same nickname (ex: {\em plus.google.com/john.dowe \/}
   and {\em facebook.com/john.dowe \/}
\end {itemize}

Similar features are in use by our approach as they provide a highly discriminatory
feature set.


\section{Web People Search using Connection Analysis}
A different approach to tackling the Web People Search problem is proposed by
\cite{connection-analysis}. Their approach bares resemblance to the approach
we propose. The main idea is to construct a linkage graph of various web pages.
For each link there exists a function which computes a confidence associated with
it. Given the confidences associated with the link, as well as all the neighboring
nodes, a cluster is established.

The clustering algorithm used is a Correlation Clustering (CC) approach.
\cite{correlation-clustering} The problem is known to be NP-Hard so the authors
have not focused on providing a better performing CC algorithm. However, they
focused on training the confidence function.
\[ f(u, v) =
    \left\{
        \begin{array}{ll}
            1 \mbox{ if } chance(u, v) \geq 0.5 \\
            0 \mbox{ otherwise}
        \end{array}
    \right.
\]
\[ chance(u, v) = probability(u,v) \mbox{ refer to the same person } \]

The difficulty lies in computing $ probability(u,v) $. Given that the training
data only gives truth values, the main factor in the computation of the
probability value is the TF/IDF metric. This approach provides an interesting
result once the clusterization is taken into account. The approach improves
on the WePS submissions by improving precision and recall by about 10\% percent
\cite{connection-analysis}.

